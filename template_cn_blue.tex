%# -*- coding:utf-8 -*-
%% start of file `template_en.tex'.
%% Copyright 2006-1008 Xavier Danaux (xdanaux@gmail.com).
%
% This work may be distributed and/or modified under the
% conditions of the LaTeX Project Public License version 1.3c,
% available at http://www.latex-project.org/lppl/.


\documentclass[11pt,a4paper]{moderncv}

\usepackage{fontspec,xunicode}
\usepackage[slantfont,boldfont]{xeCJK}
\usepackage{xcolor}  % replace by the encoding you are using

\usepackage{ifplatform} % 判断平台


% \setmainfont{Tahoma}
% \setmainfont{Times New Roman}  % 缺省英文字体.serif是有衬线字体sans serif无衬线字体

\ifwindows
  % Windows 下使用的代码
  \setCJKmainfont[ItalicFont=KaiTi, BoldFont=SimHei]{SimSun}  % 衬线字体 缺省中文字体为
  \setCJKsansfont{SimSun}
  \setmonofont{FangSong}                 % 等宽字体: FangSong, Consolas
\fi

\ifmacosx
  % macOS 下使用的代码
  \setCJKmainfont[ItalicFont={Kai}, BoldFont={Hei}]{STSong}  % 衬线字体 缺省中文字体为
  \setCJKsansfont{STSong}
  \setCJKmonofont{STFangsong}  % 中文等宽字体
\fi


%-----------------------xeCJK下设置中文字体------------------------------%
\setCJKfamilyfont{song}{SimSun}  % 宋体 song
\newcommand{\song}{\CJKfamily{song}}
\setCJKfamilyfont{fs}{FangSong_GB2312}  % 仿宋2312 fs
\newcommand{\fs}{\CJKfamily{fs}}
\setCJKfamilyfont{yh}{Microsoft YaHei}  % 微软雅黑 yh
\newcommand{\yh}{\CJKfamily{yh}}
\setCJKfamilyfont{hei}{SimHei}  % 黑体  hei
\newcommand{\hei}{\CJKfamily{hei}}
\setCJKfamilyfont{hwxh}{STXihei}  % 华文细黑  hwxh
\newcommand{\hwxh}{\CJKfamily{hwxh}}
\setCJKfamilyfont{asong}{Adobe Song Std}  % Adobe 宋体  asong
\newcommand{\asong}{\CJKfamily{asong}}
\setCJKfamilyfont{ahei}{Adobe Heiti Std}  % Adobe 黑体  ahei
\newcommand{\ahei}{\CJKfamily{ahei}}
\setCJKfamilyfont{akai}{Adobe Kaiti Std}  % Adobe 楷体  akai
\newcommand{\akai}{\CJKfamily{akai}}


%------------------------------设置字体大小------------------------%
\newcommand{\chuhao}{\fontsize{42pt}{\baselineskip}\selectfont}  % 初号
\newcommand{\xiaochuhao}{\fontsize{36pt}{\baselineskip}\selectfont}  % 小初号
\newcommand{\yihao}{\fontsize{28pt}{\baselineskip}\selectfont}  % 一号
\newcommand{\erhao}{\fontsize{21pt}{\baselineskip}\selectfont}  % 二号
\newcommand{\xiaoerhao}{\fontsize{18pt}{\baselineskip}\selectfont}  % 小二号
\newcommand{\sanhao}{\fontsize{15.75pt}{\baselineskip}\selectfont}  % 三号
\newcommand{\sihao}{\fontsize{14pt}{\baselineskip}\selectfont}  % 四号
\newcommand{\xiaosihao}{\fontsize{12pt}{\baselineskip}\selectfont}  % 小四号
\newcommand{\wuhao}{\fontsize{10.5pt}{\baselineskip}\selectfont}  % 五号
\newcommand{\subwuhao}{\fontsize{10pt}{\baselineskip}\selectfont}  % 次五号
\newcommand{\xiaowuhao}{\fontsize{9pt}{\baselineskip}\selectfont}  % 小五号
\newcommand{\liuhao}{\fontsize{7.875pt}{\baselineskip}\selectfont}  % 六号
\newcommand{\qihao}{\fontsize{5.25pt}{\baselineskip}\selectfont}  % 七号


% \usepackage{fontawesome}
% \setCJKmainfont[BoldFont={WenQuanYi Micro Hei/Bold}]{WenQuanYi Micro Hei}
% \defaultfontfeatures{Mapping=tex-text}
% \XeTeXlinebreaklocale "zh"
% \XeTeXlinebreakskip = 0pt plus 1pt minus 0.1pt
% moderncv themes
\moderncvtheme[blue]{classic}  % optional argument are 'blue' (default), 'orange', 'red', 'green', 'grey' and 'roman' (for roman fonts, instead of sans serif fonts)
% \moderncvtheme[green]{classic}  % idem
% \moderncvtheme[blue,roman]{hht}
% character encoding



% adjust the page margins
\usepackage[scale=0.9]{geometry}
% \setlength{\hintscolumnwidth}{3cm}  % if you want to change the width of the column with the dates
% \AtBeginDocument{\setlength{\maketitlenamewidth}{6cm}}  % only for the classic theme, if you want to change the width of your name placeholder (to leave more space for your address details
\AtBeginDocument{\recomputelengths}  % required when changes are made to page layout lengths

% personal data
\firstname{郭}
\familyname{冰放}
% \title{Guo BingFang}  % optional, remove the line if not wanted
% \address{杭州}{}  % optional, remove the line if not wanted
% \address{1990/11/11}{}  % optional, remove the line if not wanted
\mobile{13945718471}  % optional, remove the line if not wanted
% \fax{fax (optional)}  % optional, remove the line if not wanted
\email{g1214814536@live.com}  % optional, remove the line if not wanted
% \homepage{Blog: http://geekplux.com}  % optional, remove the line if not wanted
\social[github]{GitHub: https://github.com/greatbridf}
\extrainfo{%
  WeChat: greatbridf \\
  QQ: 1214814536
}

\photo[64pt]{avatar.jpg}  % '64pt' is the height the picture must be resized to and 'picture' is the name of the picture file; optional, remove the line if not wanted
% \quote{China\TeX 您的LaTeX乐园,TeX\&\LaTeX 王国}  % optional, remove the line if not wante

% \nopagenumbers{}  % uncomment to suppress automatic page numbering for CVs longer than one page


%----------------------------------------------------------------------------------
%            content
%----------------------------------------------------------------------------------
\begin{document}
\maketitle
\vspace*{-14mm}

\section{个人简介}
\cventry{}{来自同济大学计算机科学与技术专业的本科学生}{}{}{}{一名对计算机领域充满热情的学生,从初中开始,我在多年的学习中积累了扎实的理论基础和丰富的实践经验。我能自驱动型学习,有较强的自学能力;能够快速学习并上手掌握新技术;对新事物有强烈的好奇心,喜欢尝试新技术。在操作系统开发项目中的深入参与,使我对系统设计、内核机制以及性能优化有了深刻的理解,并培养了团队协作能力和创新思维。}
\section{教育经历}
\cventry{2022.9-}{本科26届在读}{同济大学}{计算机科学与技术}{}{}  % arguments 3 to 6 are optional
\cvlistitem{同济大学电子信息工程学院本科生奖学金-\emph{二等奖}}
\cvlistitem{2024 秋冬季R-Core开源操作系统训练营-\emph{优秀学员}}
\cvlistitem{2024年全国大学生计算机系统能力大赛-操作系统设计赛(华东区域赛)-\emph{一等奖}}
\cvlistitem{2025年全国大学生计算机系统能力大赛-操作系统设计赛(国赛)-\emph{初赛入围,决赛进行中}}
\section{实习经历}
\cvline{\textbf{暑期实习:腾讯}(2025.7-至今)}{\textbf{内核增值组}:参与 Linux 内核内存、Swap以及内存压缩、节省相关工作。\newline
调查并解决 MGLRU 在动态切换时 cgroup 的 OOM 问题,以及高内存压力下并发内存回收引起的 cgroup OOM问题,提出修复并合并,提高了系统稳定性。进行内存回收时 Batch TLB flush 相关内容的调查以及将上游 Patch 回合,提高内存回收时效率。\newline
从零开始进行内部用户态内存回收 Daemon 的 Rust 化重写,并将 cgroup 的 Rust 抽象至单独模块。实现多 cgroup 回收及 cgroup 状态读取刷新异步化、并行化,高负载场景下降低回收延迟 \textbf{50\%},日常回收延迟可降至几\textbf{ms}级别。结合同步降低回收比例,可达到平滑内存回收开销、加速突发场景下内存回收使占用率回归的目的。同时便于将来整合至内核内部,进一步消除用户与内核态通信成本,降低开销。
}
\section{项目经验}
\cvline{\textbf{gblibstdc++}}{造轮子搓的C++标准库,包括了常用的容器、算法、迭代器等。用于早期版本的Eonix内核。}
\cvline{\textbf{Eonix}}{自己写的操作系统内核,使用C++(早期)及Rust(2024年开始)开发。完成了关键模块的设计与实现,包括进程管理、内存调度和文件系统接口等,拥有自己独特的抽象及实现。此外还实现了不错的系统运行效率,并拥有较高的完整度,可以运行 gcc 并进行编译。}
\cvline{\textbf{R-Core训练营学员}}{深入学习了组件化内核的设计与实现,对现代操作系统的模块化架构有了更为全面的理解,为后续的项目开发奠定了坚实的基础。}
\cvline{\textbf{Starry-Next}}{作为项目贡献者,完善系统调用接口并尝试将其移植到宏内核架构。优化了多个系统调用的实现,增强了内核在资源管理和用户态交互方面的性能和灵活性,积极探索宏内核架构的设计与适配。}
\section{技能}
\cvline{\textbf{后端}}{熟练掌握 C++/Rust 等语言。掌握 JavaScript 及 Node 的使用。对于系统中内存管理、进程管理、文件系统等有较多的了解,掌握网络协议以及相关的编程。}
\cvline{\textbf{前端}}{掌握常见前端开发框架,如 Vue、Bootstrap、Electron 等,并拥有项目实践经验。}
\cvline{\textbf{数据库}}{掌握数据库相关的知识,熟悉 MariaDB 数据库的使用。}
\cvline{\textbf{语言}}{六级583分,熟练掌握英语的听说及阅读,以及英文文档、文章阅读能力。}
\cvline{\textbf{其他}}{熟练使用 Vim/Makefile/CMake/Git/Shell 等进行开发,熟练掌握 Linux 系统的使用及原理,云服务器的部署、服务配置。熟练使用 Docker 进行开发和部署。熟练使用调试工具进行程序调试。}

% \subsection{Vocational}
% \cventry{year--year}{Job title}{Employer}{City}{}{Description}  % arguments 3 to 6 are optional
% \cventry{year--year}{Job title}{Employer}{City}{}{Description}  % arguments 3 to 6 are optional
% \subsection{Miscellaneous}
% \cventry{year--year}{Job title}{Employer}{City}{}{Description line 1\newline{}Description line 2}% arguments 3 to 6 are optional

% \section{Languages}
% \cvlanguage{language 1}{Skill level}{Comment}
% \cvlanguage{language 2}{Skill level}{Comment}
% \cvlanguage{language 3}{Skill level}{Comment}

% \section{Computer skills}
% \cvcomputer{category 1}{XXX, YYY, ZZZ}{category 4}{XXX, YYY, ZZZ}
% \cvcomputer{category 2}{XXX, YYY, ZZZ}{category 5}{XXX, YYY, ZZZ}
% \cvcomputer{category 3}{XXX, YYY, ZZZ}{category 6}{XXX, YYY, ZZZ}

% \section{Interests}
% \cvline{篮球}{\small 体力与技巧}
% \cvline{hobby 2}{\small Description}
% \cvline{hobby 3}{\small Description}

% \renewcommand{\listitemsymbol}{-}  % change the symbol for lists

% \section{Extra 1}
% \cvlistitem{Item 1}
% \cvlistitem{Item 2}
% \cvlistitem[+]{Item 3}  % optional other symbol% XeLaTeX can use any Mac OS X font. See the setromanfont command below.
% Input to XeLaTeX is full Unicode, so Unicode characters can be typed directly into the source.

% The next lines tell TeXShop to typeset with xelatex, and to open and save the source with Unicode encoding.

% !TEX TS-program = xelatex
% !TEX encoding = UTF-8 Unicode

% \section{Extra 2}
% \cvlistdoubleitem[\Neutral]{Item 1}{Item 4}
% \cvlistdoubleitem[\Neutral]{Item 2}{Item 5}
% \cvlistdoubleitem[\Neutral]{Item 3}{}

%% Publications from a BibTeX file
% \nocite{*}
% \bibliographystyle{plain}
% \bibliography{publications}  % 'publications' is the name of a BibTeX file

% \begin{thebibliography}{}
% \bibitem[]{} 移动增强现实可视化综述[C]. ChinaVis 2017.
% \end{thebibliography}


\end{document}


%% end of file `template_en.tex'.

%%% Local Variables:
%%% mode: latex
%%% TeX-command-extra-options: "-shell-escape"
%%% TeX-master: t
%%% TeX-engine: xetex
%%% End:
